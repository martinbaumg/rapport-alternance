\documentclass[11pt, a4paper]{article}
\usepackage[french]{babel}
\usepackage{caption}
\usepackage{graphicx}
\usepackage[T1]{fontenc}
\usepackage{listings}
\usepackage{geometry}
\geometry{
  left=2.5cm,
  right=2.5cm,
  top=2.5cm,
  bottom=2.5cm,
}
\usepackage{pgfplots}

\usepgfplotslibrary{polar}
\pgfplotsset{compat=1.12} 

\newenvironment*{remerciements}{%
\renewcommand*{\abstractname}{Remerciements}
\begin{abstract}
}{\end{abstract}}

\pgfplotsset{width=10cm,compat=1.9}
\usepackage[colorlinks=true,linkcolor=black,anchorcolor=black,citecolor=black,filecolor=black,menucolor=black,runcolor=black,urlcolor=black]{hyperref}
\usepackage{fancyhdr}
\pagestyle{fancy}
\lhead{}
\rhead{}
\chead{}
\rfoot{\thepage}
\lfoot{Martin Baumgaertner}
\cfoot{}

\renewcommand{\headrulewidth}{0.4pt}
\renewcommand{\footrulewidth}{0.4pt}

\begin{document}
\begin{titlepage}
	\newcommand{\HRule}{\rule{\linewidth}{0.5mm}} 
	\center 
	\textsc{\LARGE iut de colmar}\\[4.5cm] 
	\begin{figure}
		\centering
		\includegraphics[width=0.7\textwidth]{img/uha.png}
	\end{figure}
	\textsc{\Large département réseaux et télécommunications}\\[0.5cm] 
	\textsc{\large en alternance chez orange}\\[0.5cm]
	\HRule\\[0.75cm]
	{\Large\bfseries Connectivité et Innovation : Mon Parcours d'Alternance au sein d'Orange}\\[0.4cm]
	\HRule\\[0.5cm]
	\textsc{\large\bfseries candidat : martin baumgaertner\\
	tuteur universitaire : m. albert\\
	maître de stage : m. camille }\\[6cm] 
	\begin{figure}[h]
		\centering
		\includegraphics[width=0.2\textwidth]{img/orange.png}
	\end{figure}

	\vfill\vfill\vfill
	% {\large\today} 
	\vfill
\end{titlepage}
\newpage
\begin{remerciements}
	\textit{Je tiens à exprimer ma sincère gratitude envers toute l'équipe d'Orange chez 
	laquelle j'ai eu le privilège de réaliser mon alternance. Mes remerciements
	vont particulièrement à Felix Camille et Osman Demir, qui m'ont guidé tout
	au long de cette expérience enrichissante.}\\

	\textit{Je tiens également à remercier l'ensemble de mes collègues pour leur accueil
	chaleureux et leur collaboration précieuse. Leur expertise et leur soutien ont
	grandement contribué à mon apprentissage et à mon développement professionnel.}\\

	\textit{Enfin, je souhaite exprimer ma reconnaissance envers l'équipe pédagogique de
	l'IUT de Colmar pour leur accompagnement et leur suivi tout au
	long de cette alternance.}\\

	\textit{Merci à tous ceux qui ont rendu cette expérience aussi mémorable et formatrice.}
	
\end{remerciements}
\newpage
\tableofcontents
\listoffigures
\newpage
\section{Préambule}
Je suis ravi de présenter ce rapport, fruit de mon expérience en alternance au sein d'Orange.

Orange est un leader mondial dans les télécommunications et les services mobiles.
L'entreprise, engagée dans l'innovation et la connectivité, joue un rôle majeur dans la
transformation numérique en offrant des solutions variées aux particuliers et aux entreprises.

Pendant mon alternance, j'ai travaillé au sein du département UCI EST
d'Orange. En tant que technicien d'invervention grand public, j'ai contribué aux projets de l'équipe, collaboré avec des
professionnels chevronnés et développé une compréhension approfondie des défis et des opportunités
au sein de cette entreprise.

L'engagement d'Orange envers l'excellence et l'innovation a profondément influencé ma vision
professionnelle. Ce rapport vise à refléter et à analyser les enseignements tirés de cette
expérience, mettant en lumière les aspects les plus pertinents de mon parcours au sein de
cette entreprise emblématique.\\

Au sein de mon alternance j'ai pu réaliser différentes missions
toutes plus importantes les unes que les autres. J'ai pu travailler sur des projets
d'envergure, et j'ai pu découvrir le monde de l'entreprise.
Je m'occupais principalement d'installation d'offres
Connect Pro, prestation installation experte, 
de contrôles de qualité d'installation et de dépannages.\\


\newpage
\section{Présentation de l'entreprise}
\subsection{Historique}
\subsection{Mon service}
\subsection{Organigramme}

\newpage 
\section{Mes missions}





\end{document}