\documentclass[12pt, a4paper]{article}
\usepackage[french]{babel}
\usepackage{caption}
\usepackage{graphicx}
\usepackage[T1]{fontenc}
\usepackage{listings}
\usepackage{geometry}
\usepackage{wrapfig}
\usepackage{mwe}

\usepackage{graphicx, wrapfig, calc}
\input{insbox}
\usepackage{lipsum}


\usepackage[edges]{forest}
\forestset{
  direction switch/.style={
    for tree={edge+=thick},
    where level>=1{folder, grow'=0}{for children=forked edge},
    where level=3{}{draw},
  },
}

\geometry{
  left=2.5cm,
  right=2.5cm,
  top=2.5cm,
  bottom=2.5cm,
}
\usepackage{pgfplots}



\usepackage{xcolor}
\usepackage{xcolor,rotating,epic,eepic}
\usepackage{tikz-qtree}
\usetikzlibrary{matrix}

\usepackage{xcolor,rotating,epic,eepic}
\usepackage{tikz}
\usepackage[babel=true,kerning=true]{microtype}


\usepgfplotslibrary{polar}
\pgfplotsset{compat=1.12} 

\newenvironment*{remerciements}{%
\renewcommand*{\abstractname}{Remerciements}
\begin{abstract}
}{\end{abstract}}

\pgfplotsset{width=10cm,compat=1.9}
\usepackage[colorlinks=true,linkcolor=black,anchorcolor=black,citecolor=black,filecolor=black,menucolor=black,runcolor=black,urlcolor=black]{hyperref}
\usepackage{fancyhdr}
\pagestyle{fancy}
\lhead{}
\rhead{}
\chead{}
\rfoot{\thepage}
\lfoot{}
\cfoot{}


\renewcommand{\headrulewidth}{0.4pt}
% \renewcommand{\footrulewidth}{0.4pt}

\begin{document}
\begin{titlepage}
	\newcommand{\HRule}{\rule{\linewidth}{0.5mm}} 
	\center 
	\textsc{\LARGE iut de colmar}\\[6.5cm] 
	
	\textsc{\Large département réseaux et télécommunications}\\[0.5cm] 
	\textsc{\large en alternance chez orange}\\[0.5cm]
	\HRule\\[0.75cm]
	{\Large\bfseries Connectivité et Innovation : Mon Parcours d'Alternance au sein d'Orange}\\[0.4cm]
	\HRule\\[0.5cm]
	\textsc{\large candidat : martin baumgaertner\\
	tuteur universitaire : m. albert\\
	maître de stage : m. camille }\\[5cm] 
	\begin{figure}[h]
		\centering
		\includegraphics[width=0.5\textwidth]{img/uha.png}
	\end{figure}
	\begin{figure}[h]
		\centering
		\includegraphics[width=0.1\textwidth]{img/orange.png}
	\end{figure}

	\vfill\vfill\vfill
	% {\large\today} 
	\vfill
\end{titlepage}
\newpage
\pagestyle{empty}
\begin{remerciements}
	\textit{Je tiens à exprimer ma sincère gratitude envers toute l'équipe d'Orange chez 
	laquelle j'ai eu le privilège de réaliser mon alternance. Mes remerciements
	vont particulièrement à Félix Camille et Osman Demir, qui m'ont guidé tout
	au long de cette expérience enrichissante.}\\

	\textit{Je tiens également à remercier l'ensemble de mes collègues pour leur accueil
	chaleureux et leur collaboration précieuse. Leur expertise et leur soutien ont
	grandement contribué à mon apprentissage et à mon développement professionnel.}\\

	\textit{Enfin, je souhaite exprimer ma reconnaissance envers l'équipe pédagogique de
	l'IUT de Colmar pour leur accompagnement et leur suivi tout au
	long de cette alternance.}\\

	\textit{Merci à tous ceux qui ont rendu cette expérience aussi mémorable que formatrice.}
	
\end{remerciements}
\newpage
\tableofcontents
\newpage
\listoffigures
\newpage
\pagestyle{fancy}
\section{Préambule}
Je suis ravi de présenter ce rapport, fruit de mon expérience en alternance chez Orange.\\

Orange est un leader mondial dans les télécommunications et les services mobiles.
L'entreprise, engagée dans l'innovation et la connectivité, joue un rôle majeur dans la
transformation numérique en offrant des solutions variées aux particuliers et aux entreprises.\\

Pendant mon alternance, j'ai travaillé au sein du département UCI\footnote{\vspace{-1cm}UCI EST : Unité Client Intervention} EST.
En tant que technicien d'invervention grand public, j'ai contribué aux projets de l'équipe, collaboré avec des
professionnels chevronnés et développé une compréhension approfondie des défis et des opportunités
de cette entreprise.\\

L'engagement d'Orange envers l'excellence et l'innovation a profondément influencé ma vision
professionnelle. Ce rapport vise à refléter et à analyser les enseignements tirés de cette
expérience, mettant en lumière les aspects les plus pertinents de mon parcours au sein de
cette entreprise emblématique.\\

Tout au long de mon alternance j'ai pu réaliser différentes missions
toutes plus importantes les unes que les autres. J'ai travaillé sur des projets
d'envergure, et j'ai pu découvrir le monde de l'entreprise.
Je m'occupais principalement d'installation d'offres
Connect Pro, la prestation installation experte, 
de contrôles de qualité d'installation et de dépannages.\\


Au cours de ce rapport, je vais détailler les différentes facettes de mon expérience,
en commençant par une rétrospective de l'histoire d'Orange, suivi d'une exploration
approfondie du département UCI EST et des missions cruciales auxquelles
j'ai participé.\\
Après avoir présenté ce contexte, je partagerai une analyse approfondie
de mon expérience, en mettant en évidence les enseignements tirés et
les compétences développées. Cette réflexion sera étayée par des exemples
concrets de missions spécifiques, illustrant ainsi la manière dont mon
alternance chez Orange a façonné mon expérience professionnelle.\\
Le rapport se terminera par une synthèse de mes contributions à
l'innovation et à la stratégie globale d'Orange, ainsi que par un
bilan personnel où je partagerai mes réflexions sur cette expérience
enrichissante et les perspectives qu'elle ouvre pour l'avenir.\\
Ainsi, ce rapport sera une plongée détaillée dans mon parcours
d'alternance chez Orange.


\newpage
\section{Présentation de l'entreprise}
\subsection{Historique}
L'histoire d'Orange est emblématique en France, ayant évolué au gré des
innovations technologiques et des réformes structurelles. Originellement connue
sous le nom des PTT (Postes, Télégraphes et Téléphones), 
l'entreprise a été le pilier des services de communication en
France. En 1990, les PTT ont été divisés en deux entités distinctes : La Poste et France Télécom.
C'est d'ailleurs pour cette raison que partout en France, les bureaux de postes 
sont très souvent situés à côtés de centraux téléphoniques d'Orange.\\

A la date du 1er juillet 2013, France Télécom deviendra Orange, et ce dans le monde entier.
Ainsi, nous pouvons retrouver Orange dans pas moins de 26 pays à travers le monde et 
compte près de plus de 140 000 collaborateurs.\\

Le siège social d'Orange est situé à Issy-les-Moulineaux,
et est dirigé par Christel Heydemann. Pour ma part, 
j'ai effectué mon alternance en étant rattaché 
à l'agence de Schiltigheim. Nous pouvons retrouver 
dans ces bureaux de nombreux collaborateurs et 
plusieurs équipes, notamment des chargés d'affaires
et des techniciens d'intervention. Au sein de l'UCI EST,
nous retrouvons plusieurs équipes d'interventions :\\

\begin{itemize}
	\item Equipe d'invervention sur le réseau structurant 
	\item Equipe d'intervention entreprise
	\item Equipe d'intervention grand public\\
\end{itemize}

Pour ma part, j'ai effectué mon alternance en tant que
technicien d'intervention grand public dans l'équipe
d'Osman Demir. 


\subsection{Mon service}
Comme décrit précédemment, j'ai effectué mon alternance
au dans l'équipe d'intervention grand public 
du bas-rhin. En tant que technicien d'intervention grand public,
nous pouvions êtres amenés à réaliser des missions diverses et 
variées. Nous étions chargés des 
expertises, des installations, des dépannages et 
des contrôles de qualité.\\

Le but d'une expertise FTTH\footnote{FTTH : Fiber To The Home} est de se rendre chez un client
qui a déjà eu plusieurs rendez-vous SAV et qui n'a pas eu de solution. 
Notre objectif est alors d'en apporter une et que l'on 
soit le dernier technicien à se rendre chez le client.\\

Les installations que nous sommes amenés à réaliser sont
la mise en service de l'offre Connect Pro d'Orange et 
la prestation installation experte (qui seront détaillées dans la partie suivante).\\


\newpage
\subsection{Organigramme}
\begin{figure}[h]
	\centering
	\begin{forest}
		direction switch
		[\textbf{Équipe grand public 67 Resp. Thomas Mall}
		[Responsables d'équipe
			[Osman Demir
			[Michael Forest]
			[Cédric Berron]
			]
		]
		[Techniciens
			[\textbf{Martin Baumgaertner}]
			[Camille Félix]
			[Pascal Bonney]
			[Brahim Bourouaieh]
			[Philippe Caspar]
			[Alexis De Sousa]
			[Julien Hecklen]
			[Arzu Karakas]
			[Moudy Toure]
			[Frédéric Vaz]
		]
		]
		]
	\end{forest}
	\caption{L'équipe GP 67}
\end{figure}
Nous pouvons voir ci-dessus que je me trouve dans 
une équipe de 10 techniciens d'intervention grand public, 
couvrant tout le département du bas-rhin. Nous sommes 
la plupart du temps répartis par zones géographiques. 
Par exemple Félix Camille est du secteur de Hagenau, tandis 
que je suis du secteur de Strasbourg. Être réparti par 
secteur ne signifie pas que nous sommes limités à Strasbourg 
par exemple. Je peux très bien être le matin à Strasbourg 
et l'après-midi à Molsheim.\\

\newpage 
\section{Mes missions}
\subsection{Introduction}
Au cours de mon alternance de 2 ans. J'ai eu la chance
de pouvoir effectuer des missions riches en 
enseignements, avec toujours des cas différents. 
Ma première année s'est déroulée en doublure avec 
mon tuteur Félix. Il m'a fait monter en compétences
et m'a permis de découvrir le monde de l'entreprise.
C'est grâce à lui que j'ai pu acquérir le
nécessaires pour être autonome. Suite à cette première
année, j'ai pu réaliser des missions seul, qui ont aussi 
été très enrichissantes. Car même si Félix m'a appris 
beaucoup de choses, il y a toujours des cas complexes 
sur lesquels nous n'étions jamais tombé en 1 an et
qui nécessitent de la réflexion. Grâce à son soutien 
téléphonique en cas de besoin, j'ai pu alors 
devenir de plus en plus autonome. 

\subsection{Ma première installation}
A mon arrivée dans l'équipe, et mes débuts chez Orange, 
je me rappelerai toujours de ma première installation 
avec Félix. C'était dans une mairie dans le nord
de l'Alsace qui avait la particularité d'avoir 
souscrit à 2 offres Connect Pro (pour deux services différents) 
alors qu'ils étaient dans les mêmes locaux. Je me rappelle que Félix m'avait
montré comment faire la première puis, il m'avait laissé
faire la seconde. Assez facile me direz-vous, mais
c'était sans compter sur le stress de découvrir 
une nouvelle solution, avec un "process"\footnote{Process : terme utilisé chez Orange pour désigner une procédure spécifique à suivre} que je ne connaissais 
pas. C'était les débuts de l'apprentissage. 

\subsection{Connect Pro}
\subsubsection{Présentation}
Connect Pro d'Orange représente une solution
de téléphonie IP novatrice, spécialement conçue pour
les petites entreprises. Elle offre la flexibilité et
la facilité d'utilisation dont ces entreprises ont
besoin pour gérer leurs communications de manière
efficace, sans la nécessité d'investir dans un
serveur IPBX physique sur site. Tout est géré de
manière transparente dans le cloud, grâce aux
serveurs d'Orange, à partir de la Livebox
que nous installons.\\

Ce qui distingue cette offre, c'est sa
personnalisation approfondie. Les clients ont
la possibilité de créer des messages personnalisés 
fait par un studio d'enregistrement professionnel 
pour accueillir leurs appelants, des prédécrochés
informatifs, des messages d'attente professionnels
et des répondeurs adaptés à leurs besoins spécifiques.
Cela leurs permet de fournir une expérience de
communication personnalisée à leurs clients
et partenaires.\\

Mais ce n'est pas tout, car la personnalisation
va encore plus loin. Par exemple, nous
pouvons paramétrer un envoi automatique de
mails à chaque fois qu'un message est déposé sur
le répondeur. Ces mails peuvent contenir le
message vocal en pièce jointe, ce qui
facilite la gestion et le suivi des messages importants.\\

\newpage
\begingroup
\setlength{\intextsep}{0pt}
\begin{wrapfigure}[20]{R}{6.5cm}
	\includegraphics[scale=.5]{img/mittel.jpg}
	\caption{Poste Mittel 67i}
\end{wrapfigure}
De plus, Connect Pro propose une variété
d'options en matière de téléphones, que
ce soit des téléphones sans fil pour
une mobilité accrue, des téléphones
filaires pour une utilisation plus
traditionnelle ou bien même des softphone 
sur PC ou smartphone disponibles avec 
l'application Cisco Webex. Cela signifie que nous pouvons
adapter la solution de téléphonie exactement aux besoins
des clients et à son environnement de travail.
Ci-contre nous pouvons voir un poste Mittel 67i
qui est un poste filaire, ainsi qu'un poste sans fil
Yealink W76P.\\
\begin{wrapfigure}[20]{R}{6.5cm}
	\includegraphics[scale=.6]{img/yealink.png}
	\caption{Yealink W76P}
\end{wrapfigure}

Bien sûr, il est important de noter que l'offre
Connect Pro d'Orange propose une flexibilité en
fonction de la technologie d'accès disponible dans
la zone du client. Pour les entreprises situées dans des
zones couvertes par la fibre optique, l'offre
"Connect Pro Fibre" offre la possibilité de
bénéficier de jusqu'à 20 lignes téléphoniques.\\

Cependant, il est important de noter qu'il
existe également une version de l'offre Connect Pro
pour les entreprises situées dans des zones
couvertes par le cuivre. L'offre "Connect Pro Cuivre"
est disponible dans ces régions, mais elle est
généralement limitée en termes de capacité et de
fonctionnalités par rapport à la version fibre.
En général, les entreprises optent pour l'offre
"Connect Pro Cuivre" uniquement lorsqu'elles ont
des besoins de communication très basiques, tels
que 1 ou 2 postes, car les réseaux
cuivre ne permettent pas une utilisation aussi
étendue ni des fonctionnalités aussi avancées 
que la fibre.\\

En fin de compte, le choix entre "Connect Pro
Fibre" et "Connect Pro Cuivre" dépendra de la
disponibilité de la technologie dans votre région,
ainsi que de vos besoins en matière de communication.
Les entreprises ayant des besoins plus importants
et souhaitant exploiter pleinement les fonctionnalités
offertes par "Connect Pro" opteront généralement
pour la fibre, tandis que celles situées dans des
zones couvertes par le cuivre et ayant des besoins
plus modestes pourront envisager la version cuivre
de l'offre.
\par\endgroup

\newpage
\subsubsection{Installations Connect Pro}
Lors de l'installation d'offres Connect Pro, nous
avons un process à suivre. Tout d'abord, nous devons
vérifier le dossier client afin de s'assurer 
qu'il n'y ait pas d'erreurs. Nous contrôlons 
le numéro de téléphone, et surtout, les adresses 
mac des postes téléphoniques. En effet, il est
important de vérifier ce paramètre car si elles 
sont erronées, les postes n'auront pas 
la bonne configuration.\\

Une fois la vérification effectuée, nous pouvons
commencer l'installation. Nous commençons généralement 
par installer la Livebox en important les différents 
paramètres de configuration personnalisé que le client 
aurait pu faire.\\

Ensuite, nous installons les postes téléphoniques
en les connectant en RJ45 à la Livebox. Une fois
les postes installés, nous pouvons alors les
configurer. Pour cela, nous nous rendons sur 
l'interface web suivante : \url{https://connect.pro.orange.fr/}
et nous nous connectons avec les identifiants
du client.\\

Nous arrivons alors sur l'interface de gestion
de l'offre Connect Pro : 
\begin{figure}[h]
	\centering
	\includegraphics[scale=1.3]{img/accueil.png}
	\caption{Accueil de l'interface de gestion}
\end{figure}

Nous retrouvons alors tous nos postes 
dans la partie "Gestion de lignes". Nous pouvons
les configurer selon les besoins de client. 

\begin{figure}[h]
	\centering
	\includegraphics[scale=0.65]{img/configuration.jpg}
	\caption{Configuration des postes}
\end{figure}

\newpage
\subsubsection{Tests et vérifications}
Une fois l'installation terminée, nous devons
vérifier que la programmation de la téléphonie 
soit correcte. Pour cela, il suffit de faire 
un appel entrant et s'assurer que les différents
messages, les renvois vers d'autres postes et
les répondeurs fonctionnent correctement. 
Nous nous assurons également qu'en cas 
de coupure de courant, tous les appels entrants
soient redirigés vers un numéro de secours.\\

Cette phase de tests est cruciale car elle permet 
de détecter les éventuelles erreurs ou anomalies
avant de présenter et de former le client 
à sa nouvelle solution de téléphonie.\\

\subsubsection{Difficultés rencontrées}
La plupart des difficultés rencontrées lors des installations
Connect Pro étaient liées aux équipements clients présents.
Parmi les principaux défis opérationnels que j'ai dû surmonter,
trois se sont révélés particulièrement prépondérants. 
Comme le fait que certains câbles RJ45 étaient mal identifiés, ce qui a entraîné
des retards et des erreurs potentielles.\\

Dans certains cas, les installations présentaient un enchevêtrement
complexe de câbles, compliquant la compréhension du réseau.
J'essayais toujours de simplifier ces installations en retirant les câbles
inutiles et en ordonnant les câbles restants.\\

Mais aussi, il m'est arrivé plusieurs fois d'être face à de vieux bâtiments 
manquants drastiquement de prises électriques pour alimenter les équipements
nécessaires. Il fallait alors essayer de trouver des solutions 
provisoires, comme la mise en 
place d'enrouleurs multiprise le temps que le client fasse les travaux nécessaires. 

Malgré ces défis, une approche proactive, une communication efficace
avec les clients ont permis de mener à
bien les installations Connect Pro avec succès. Ces
expériences m'ont permis de développer des compétences essentielles
en résolution de problèmes tout en
renforçant ma capacité à travailler dans des environnements
opérationnels complexes.

\newpage
\subsubsection{Ma contribution à l'innovation}
Au cours de toutes les installations effectuées, 
je me suis souvent rendu compte que les clients
voulaient souvent mettre en place des messages 
de répondeur temporaire pour les périodes
de fermetures exceptionnelles. Cependant,
la méthode est souvent fastidieuse et
complexe pour les clients, et ils oublient 
souvent. C'est pourquoi j'ai eu l'idée de
créer une maquette explicative pour les clients
afin qu'ils gardent une trace de la procédure. 
La voici ci-dessous, et en grand format en annexe.\\
\begin{figure}[h]
	\centering
	\includegraphics[scale=0.3]{img/maquette.png}
	\caption{Maquette explicative}
\end{figure}
\newpage


\subsection{Installations expertes}
\subsubsection{Présentation}
\subsubsection{La prestation installation experte}






\subsection{Ma mission chez Wrigley}
\section{Annexes}
\end{document}